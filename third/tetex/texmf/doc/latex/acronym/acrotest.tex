%%
%% This is file `acrotest.tex',
%% generated with the docstrip utility.
%%
%% The original source files were:
%%
%% acronym.dtx  (with options: `acrotest')
%% 
%% Copyright (C) 1994 by Tobias Oetiker.
%% All rights reserved.
%% 
%%  This package is distributed in the hope that it will be useful,
%%  but WITHOUT ANY WARRANTY; without even the implied warranty of
%%  MERCHANTABILITY or FITNESS FOR A PARTICULAR PURPOSE.
%% 
\def\filename{acronym}
\def\fileversion{v1.3}
\def\filedate{1996/09/19}
\def\docdate {94/07/13}
\documentclass{article}
\usepackage{acronym}
\begin{document}
\section{Intro}
In the early nineties, \acs{GSM} was deployed in many European
countries. \ac{GSM} offered for the first time international
roaming for mobile subscribers. The \acs{GSM}'s use of \ac{TDMA} as
its communication standard was debated at length. And every now
and then there are big discussion whether \ac{CDMA} should have
been chosen over \ac{TDMA}.

If you want to know more about \acf{GSM}, \acf{TDMA}, \acf{CDMA}
and \ac{oa}, just read a book about mobile communication.
\section{Acronyms}
\begin{acronym}
 \acro{GSM}{Global System for Mobile communication}.
      \acs{GSM} is the new standard for digital cellular
      communication in Europe.
 \acro{TDMA}{Time Division Multiple Access}.
      Some would say, that this is not as good as \ac{CDMA}.
 \acro{CDMA}{Code Division Multiple Access}. The spread spectrum
      modulation used in the Qualcomm system.
 \acrodef{oa}{other acronyms}
\end{acronym}
\end{document}
\endinput
%%
%% End of file `acrotest.tex'.

%%
%% This is file `acrotest.tex',
%% generated with the docstrip utility.
%%
%% The original source files were:
%%
%% acronym.dtx  (with options: `acrotest')
%% 
%% Copyright (C) 1994,1999 by Tobias Oetiker.
%% All rights reserved.
%% 
%%  This package is distributed in the hope that it will be useful,
%%  but WITHOUT ANY WARRANTY; without even the implied warranty of
%%  MERCHANTABILITY or FITNESS FOR A PARTICULAR PURPOSE.
%% 
%%  It may be distributed under the conditions of the LaTeX Project Public
%%  License, either version 1.2 of this license or (at your option) any
%%  later version.  The latest version of this license is in
%%     http://www.latex-project.org/lppl.txt
%%  and version 1.2 or later is part of all distributions of LaTeX
%%  version 1999/12/01 or later.
%% 
%%  The list of all files belonging to the `acronym' package is
%%  given in the file `readme'.
%% 
\def\filename{acronym}
\def\fileversion{v1.6}
\def\filedate{2000/05/21}
\def\docdate {2000/05/21}
\documentclass{article}
\usepackage{acronym}
\begin{document}
\section{Intro}
In the early nineties, \acs{GSM} was deployed in many European
countries. \ac{GSM} offered for the first time international
roaming for mobile subscribers. The \acs{GSM}'s use of \ac{TDMA} as
its communication standard was debated at length. And every now
and then there are big discussion whether \ac{CDMA} should have
been chosen over \ac{TDMA}.

If you want to know more about \acf{GSM}, \acf{TDMA}, \acf{CDMA}
and \ac{oa}, just read a book about mobile communication.

\subsection{Some chemistry and physics}
\ac{NAD+} is a major electron acceptor in the oxidation of fuel
molecules. The reactive part of \ac{NAD+} is its nictinamide ring,
a pyridine derivate.

One mol consists of \acs{NA} atoms or molecules. There is a relation
between the constant of Boltzmann and the \acl{NA}:
\begin{equation}
  k = R/\acs{NA}
\end{equation}

\section{Acronyms}
\begin{acronym}
 \acro{GSM}{Global System for Mobile communication}.
      \acs{GSM} is the new standard for digital cellular
      communication in Europe.
 \acro{TDMA}{Time Division Multiple Access}.
      Some would say, that this is not as good as \ac{CDMA}.
 \acro{CDMA}{Code Division Multiple Access}. The spread spectrum
      modulation used in the Qualcomm system.
 \acrodef{oa}{other acronyms}
 \acro{NAD+}[NAD\textsuperscript{+}]{Nicotinamide Adenine Dinucleotide}.
 \acro{NA}[\ensuremath{N_{\mathrm A}}]{Number of Avogadro}:
        $\acs{NA} = 6.022045\cdot10^{23}\,\mathrm{mol}^{-1}$
\end{acronym}
\end{document}
\endinput
%%
%% End of file `acrotest.tex'.

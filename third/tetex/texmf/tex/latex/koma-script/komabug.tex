%%
%% @ is a letter
%%
\catcode`\@=11

%%
%% Grab the initex file list
%%
%% If this file is called via
%%     latex "\input{latexbug}" or some
%% similar command sequence rather than
%%     latex latexbug
%% then the debugging info in \reserved@a will already have been lost.
%% This might not matter, but if it does we may ask the user to resubmit
%% the report.
\ifx\reserved@b\@undefined
  \ifx\reserved@a\@gobble
    \def\@inputfiles{NONE}
  \else
    \let\@inputfiles\reserved@a
  \fi
\else
  \def\@inputfiles{LOST}
\fi

%%
%% Output stream to produce the bug report template.
%%
\newwrite\msg
\immediate\openout\msg=\jobname.msg


%%
%% Check that LaTeX2e is being used.
%%
\ifx\undefined\newcommand
 \newlinechar`\^^J%
 \immediate\write17{^^J%
    You must use LaTeX2e to generate the bug report!^^J^^J%
    Sie muessen LaTeX2e verwenden, um die Fehlermeldung zu erzeugen!}%

 \let\relax\end
\else
 \def\@tempa{LaTeX2e}\ifx\@tempa\fmtname\else
  \immediate\write17{^^J%
   You must use LaTeX2e to generate the bug report!^^J^^J%
   Aeltere Versionen von LaTeX werden nicht unterstuetzt.^^J%
   Sie muessen LaTeX2e verwenden, um de Fehlermeldung zu erzeugen!}%
  \let\relax\@@end
\fi\fi

%%
%% \wmsg writes to the terminal, and the .msg file
%% \wmsg* just writes to the .msg file
%% \typeout just writes to the terminal
%%

\def\wmsg{\bgroup\@ifstar{\interactivefalse\@wmsg}\@wmsg}


\relax
\endlinechar=-1

\def\@wmsg#1{%
  \ifinteractive\immediate\write17{#1}\fi
  \immediate\write\msg{#1}%
  \egroup}

%%
%% if \interactivefalse just make a blank template.
%%
\newif\ifinteractive
\interactivetrue

%%
%% Prompt for an answer from the user, if the answer is not
%% provided by the cfg file.
%%

\def\readifnotknown#1{%
 \@ifundefined{#1}%
    {{\message{#1> }%
     \catcode`\^^I=12 \let\do\@makeother\dospecials
     \global\read\m@ne t\expandafter o\csname#1\endcsname}}%
    {\message{\csname#1\endcsname}}}

%%
%% Pause so messages do not scroll off screen.
%%
\def\pause{%
  \ifinteractive
   \message{Mit der <Return>-Taste geht es weiter. }%
   \read\m@ne to \@tempa
  \fi}

%%
%% Try to get Version
%%
\def\GetVersion#1{
 \def\ProvidesFile##1{\@ifnextchar [{\@P@F}{\@P@F[1996/10/31 ]}}
 \def\@P@F[##1 ##2]{\xdef\komaversion{##1}\csname endinput\endcsname}
 \InputIfFileExists{#1}
  {}
  {\def\noscrclass{%
   ! Die Datei `#1' konnte nicht gefunden werden!^^J%
   ! Zur Bestimmung der aktuellen Version ist es unbedingt^^J%
   ! erforderlich, dass diese Datei sich im selben Verzeichnis^^J%
   ! wie `komabug.tex' befindet!}
   \errmessage{\noscrclass}
   \errhelp{Beenden Sie TeX mit `x' und starten Sie komabug neu}}}

%%
%% Opening Banner.
%%
\typeout{^^J%
============================================================^^J%
^^J%
KOMA-Script Fehlermeldungsgenerator^^J%
===================================^^J%
Die Bearbeitung dieser Datei mit LaTeX erzeugt das Formular \jobname.msg,^^J
um Fehlermeldungen zum KOMA-Script-Paket zu melden.^^J^^J%
* Schreiben Sie Ihre Meldung nach Moeglichkeit in Deutsch.^^J \space
  Notfalls ist auch Englisch moeglich.^^J%
* Bitte fassen Sie sich kurz.^^J%
* Bitte halten Sie keine Information zurueck, die moeglicherweise^^J \space
  wichtig sein koennte.^^J}


\InputIfFileExists{komabug.cfg}
                  {\typeout{** komabug.cfg wird verwendet **}}
                  {}

\count@=0

\typeout{%
Verschiedene Bereiche werden von diesem Generator unterstuetzt, die^^J%
sich auf verschiedene Dateien im KOMA-Script-Paket beziehen:^^J^^J
0) komascr:\space\@spaces
         Auspacken und Installieren von KOMA-Script.^^J
1) scrguide:\@spaces
         Die KOMA-Script-Anleitung.^^J
2) scrclass:\@spaces
         Die KOMA-Script-Standard-Klassen.^^J
3) scrlettr:\@spaces
         Die KOMA-Script-Brief-Klasse und die zugehoerigen Pakete.^^J
4) typearea:\@spaces
         Automatische und individuelle Satzspiegelberechnung.^^J
5) scrpage:\space\@spaces
         Individuelle Kopf- und Fusszeilendefinition.^^J
6) scrtime:\space\@spaces
         Kleine Zugaben fuer Datums- und Zeitinformationen.^^J%
}
\message{Bitte waehlen Sie einen der Bereiche von 0--6 aus:  }
\read\m@ne to \answer
\count@=\answer\relax

\ifcase\count@
 \def\category{komascr}\GetVersion{scrclass.dtx}
 \let\categoryversion\komaversion
\or
 \def\category{scrguide}\GetVersion{scrguide.tex}
 \let\categoryversion\komaversion\GetVersion{scrclass.dtx}
\or
 \def\category{scrclass}\GetVersion{scrclass.dtx}
 \let\categoryversion\komaversion
\or
 \def\category{scrlettr}\GetVersion{scrlettr.dtx}
 \let\categoryversion\komaversion\GetVersion{scrclass.dtx}
\or
 \def\category{typearea}\GetVersion{scrclass.dtx}
 \let\categoryversion\komaversion
\or
 \def\category{scrpage}\GetVersion{scrpage.dtx}
 \let\categoryversion\komaversion\GetVersion{scrclass.dtx}
\or
 \def\category{scrtime}\GetVersion{scrtime.dtx}
 \let\categoryversion\komaversion\GetVersion{scrclass.dtx}
\else
 \errhelp{Beenden Sie TeX mit `x' und starten Sie komabug neu}
 \def\badcategory{Nur die Bereiche 0,...,6 werden derzeit unterstuetzt}
 \errmessage{\badcategory}
\fi


\typeout{^^J%
============================================================^^J%
^^J%
Bitte eine einzeilige (< 50 Zeichen) Beschreibung des Problems.%
^^J^^J%
Wenn Sie fuer die Meldung eMail verwenden, setzen Sie diese Beschreibung^^J
bitte auch als Betreff (`Subject') der eMail ein:%
^^J \@spaces\@spaces\space
                 |<------------------------------------------------>|}


\loop
\let\synopsis\relax
\readifnotknown{synopsis}
\ifx\synopsis\@empty
\repeat


\typeout{%
^^J%
Dieser Generator kann auf zwei Arten verwendet werden.^^J%
Im interaktiven Betrieb, werden Ihnen Fragen zur direkten Beantwortung^^J%
gestellt. Ansonsten wird ein leeres Formular erzeugt, das Sie dann mit^^J%
einem Editor ausfuellen muessen.^^J%
}

\typeout{Interaktive Sitzung (j/n)? }
\readifnotknown{interactive}

\ifx\interactive\@empty
   \def\interactive{n}
\fi

%%
%% Allow anything begining with `j' or `J' for yes.
%%
\edef\interactive{\uccode`\expandafter\@car\interactive\@nil}
\ifnum \interactive=`J \else\interactivefalse\fi


%%
%% Header in the msg file.
%%
\wmsg*{^^J%
 KOMA-Fehlermeldung.^^J%
\ifnum \interactive=`Y Interaktiv \else Formular \fi
 erzeugt mit komabug.tex am
 \space\number\year-\two@digits\month-\two@digits\day^^J%
^^J%
 Meldungen bis 32 KByte koennen per eMail im MausNet an^^J%
 Markus Kohm @ KA2 verschickt werden.^^J%
 Die Meldung kann auch per eMail an unlf@rz.uni-karlsruhe.de^^J%
 verschickt werden.^^J%
 Bitte verwenden Sie dann als Betreff:^^J%
 \space KOMA-BUG:\space\synopsis^^J%
 ============================================================^^J}


%%
%% Category of bug, obtained earlier but put out now, after the header.
%%
\wmsg{>Bereich: \category}
\wmsg{>Version: \categoryversion}

%%
%% synopsis of bug, obtained earlier but put out now, after the header.
%%
\wmsg{>Betreff: \synopsis}

\begingroup
 \global\let\format\@empty
 \gdef\hyphenation{standard}
 \def\immediate#1#{\xdef\hyphenation}
 \def\typeout#1{\xdef\format{\format#1}}
 \the\everyjob
\endgroup

\wmsg{>Format: \format}

\wmsg{>KOMA-Script: \komaversion}

\ifinteractive
%%
%% if interactive, \wread reads a line (verbatim) and write it to the
%% .msg file, until a blank line is entered.
%%
  \def\wread{{\catcode`\^^I=12
  \let\do\@makeother\dospecials
  \let\lastanswer\answer
  \message{=> }\read\m@ne to \answer
  \ifx\lastanswer\@empty
    \let\lastanswer\answer
  \fi
  \ifx\lastanswer\@empty
  \else
    \immediate\write\msg{\answer}
    \expandafter\wread
  \fi}}
\else
%%
%% If non-interactive, \wread just writes a blank line to the .msg file,
%% and \wmsg does not write to the terminal.
%%
  \def\wread{\wmsg{}}
\fi

%%
%% \copytomsg copies the contents of a file into the .msg file.
%% (at least it does it as well as TeX can, so there may be
%% transcription problems with 8-bit characters).
%%
%% It does a line count, and complains if the test file is
%% too large.

\chardef\inputfile=15

\newcount\linecount

\def\copytomsg#1{{%
   \def\do##1{\catcode`##1=11}%
   \dospecials
   \global\linecount\z@
   \openin\inputfile#1\relax
   \def\thefile{#1}%
   \@copytomsg
   \closein\inputfile}}

\def\@copytomsg{%
   \ifeof\inputfile
      \typeout{*** \thefile\space Zeilen = \the\linecount}
   \else
      \global\advance\linecount\@ne
      \read\inputfile to \inputline
      \ifx\inputline\@empty
         \wmsg*{}
      \else
         \wmsg*{\inputline}
      \fi
      \expandafter\@copytomsg
   \fi}


%%
%% Test the age of the current format.
%%
\def\getage#1/#2/#3\@nil{%
  \count@\year
  \advance\count@-#1\relax
  \multiply\count@ by 12\relax
  \advance\count@\month
  \advance\count@-#2\relax}
%
\expandafter\getage\fmtversion\@nil
%%
%% \count@ should now be the age of the format in months.
%%
\ifnum\count@>12
\def\oldformat{^^J%
   ! Ihre LaTeX-Installation ist ueber ein Jahr alt.^^J%
   ! Bitte denken Sie ueber ein Update nach, ehe Sie diese Meldung^^J%
   ! abschicken.^^J%
   ! Vergleichen Sie wenigstens das Datum des Paketes mit dem Datum^^J%
   ! der LaTeX-Version. LaTeX sollte nicht mehr als 6 Monate aelter^^J%
   ! als KOMA-Script sein. Ansonsten koennte der Fehler in einer^^J%
   ! Unverstraeglichkeit zwischen Format- und Paketversion liegen.}
%%
%% Put the message in a macro to improve the look of the error mesage.
%%
%
\errhelp{Wenn Sie dennoch fortfahren wollen, druecken Sie einfach <Return>.}
\errmessage{\oldformat}
\fi
%
\expandafter\getage\komaversion\@nil
\ifnum\count@>12
\def\oldversion{^^J%
   ! Ihre KOMA-Script-Version ist ueber ein Jahr alt.^^J%
   ! Bitte denken Sie ueber ein Update nach, ehe Sie diese Meldung^^J%
   ! abschicken.^^J%
   ! Vergleichen Sie wenigstens das Datum des Paketes mit dem Datum^^J%
   ! der LaTeX-Version. LaTeX sollte nicht mehr als 6 Monate aelter^^J%
   ! oder juenger als KOMA-Script sein. Ansonsten koennte der Fehler^^J%
   ! in einer Unverstraeglichkeit zwischen Format- und Paketversion^^J%
   ! liegen.}
\errhelp{Wenn Sie dennoch fortfahren wollen, druecken Sie einfach <Return>.}
\errmessage{\oldversion}
\fi


%%
%% Now use \wmsg and \wread for each of the multi-line fields
%% in the form. Currently one-line fields use \read directly.
%%
\ifinteractive
  \typeout{^^JIhr Name:}
  \readifnotknown{name}
\else
  \ifx\name\undefined
    \def\name{ < GEBEN SIE IHREN NAMEN EIN > }
  \fi
\fi


\ifinteractive
  \typeout{^^JIhre Addresse (eMail bevorzugt):}
  \readifnotknown{address}
\else
  \ifx\address\undefined
    \def\address{ < GEBEN SIE IHRE (EMAIL-)ADRESSE EIN > }
  \fi
\fi


\ifinteractive
  \typeout{^^JDas verwendete Computersystem (z. B. Atari, Linux, Mac):}
  \readifnotknown{computersystem}
\else
  \ifx\computersystem\undefined
    \def\computersystem{ < GEBEN SIE IHR VERWENDETES COMPUTERSYSTEM EIN > }
  \fi
\fi


\wmsg*{>Adresse: \name\space<\address>}

%%
%% >Organisation: is really a GNATS multiline filed
%% but we treat it as a one-line field.
%%
\wmsg*{>Organisation: \ifx\organisation\undefined
                        \ifx\organization\undefined\else
                           \organization
                        \fi
                       \else
                         \organisation
                       \fi}


%%
%% Test which format is being used. These fields are completed
%% automatically even if the blank template is being produced.
%%

\wmsg*{>Voraussetzungen:}
\wmsg*{ \string\@TeXversion: \meaning\@TeXversion
        \ifx\@TeXversion\@@undefined
         \space (Standardeinstellung fuer TeX3.141 und spaeter)\fi}
\wmsg*{ \string\@currdir: \meaning\@currdir}
\wmsg*{ \string\input@path: \meaning\input@path
        \ifx\input@path\@@undefined
         \space (Standardeinstellung)\fi}
\wmsg*{ System: \computersystem}

\wmsg*{>Beschreibung:}
\ifinteractive
  \typeout{%
    Beschreibung des Problems:^^J^^J%
    \@spaces Die Beantwortung dieser Frage kann mehrere Zeilen^^J%
    \@spaces einnehmen (jede wird durch die Eingabeaufforderung^^J%
    \@spaces => eingeleitet).^^J%
    \@spaces Durch zwei aufeinanderfolgende Leerzeilen wird die^^J%
    \@spaces Antwort beenden.}
\else
\wmsg{ < GEBEN SIE HIER IHRE PROBLEMBESCHREIBUNG EIN >}
\fi
\wread


%%
%% insertion of the test file
%%
\ifinteractive
   \typeout{^^J%
 Name einer KURZEN, SELBSTERKLAERENDEN Datei, bei der das Problem^^J%
 auftritt (die Datei sollte wirklich so kurz wie moeglich sein,^^J%
 nicht mehr als 60 Zeilen):^^J^^J%
 Damit LaTeX diese Datei einlesen kann, geben Sie bitte den kompletten^^J%
 Namen einschliesslich des Verzeichnisses an, falls die Datei nicht im^^J%
 aktuellen Verzeichnis zu finden ist.^^J^^J%
 Falls Sie keinen Fehler in einer der Klassen oder Pakete melden und^^J%
 daher keine Testdatei existiert, druecken Sie einfach <Return>}
   \message{filename> }\read\m@ne to \filename
\else
   \def\filename{}
\fi

%%
%% Try to find the .tex file and .log file
%%


\ifx\filename\@empty
 \ifinteractive
  \typeout{^^J^^JKeine Testdatei.^^J^^J%
   Drei Arten von Meldungen werden unterstuetzt:^^J^^J%
    0) SW-Fehler:^^J\@spaces
       Software-Fehler, unbedingt eine Testdatei beilegen!^^J
    1) DOC-Fehler:^^J\@spaces
       Fehler in oder unverstaendliche Anleitung.^^J
    2) Aenderungswunsch:^^J\@spaces
       Kein Fehler sondern eine Frage nach Aenderund oder Hilfe.^^J}
  \message{Bitte waehlen Sie eine Art von 0--2:  }
  \read\m@ne to \answer
  \ifx\answer\@empty
    \def\answer{-1}
  \fi
  \count@=\answer\relax
 \else
  \count@=\z@
 \fi
 \ifcase\count@
  \ifinteractive\wmsg{>Unterbereich: SW-Fehler}\fi
  \typeout{^^J! Bitte ergaenzen Sie Beispiel- und LOG-Datei in der Meldung!}
  \pause
  \wmsg*{%
   Beispieldatei, die das Problem verdeutlicht:^^J%
   ============================================^^J%
   \space< HIER TESTDATEI EINFUEGEN >^^J%
   ^^J%
   LOG-Datei vom LaTeX-Lauf der Beispieldatei:^^J%
   ===========================================^^J%
   \space< HIER LOG ZUR TESTDATEI EINFUEGEN >}
 \or
  \wmsg{>Unterbereich: DOC-Fehler}
 \or
  \wmsg{>Unterbereich: Aenderungswunsch}
 \else
   \errhelp{Beenden Sie TeX mit `x' und starten Sie komabug neu}
   \def\badcategory{Nur die Bereiche 0, 1, 2 werden derzeit unterstuetzt}
   \errmessage{\badcategory}
 \fi

\else

\filename@parse\filename

\IfFileExists{\filename}{\edef\samplefile{\filename}}{}

\IfFileExists{\filename@area\filename@base.log}
  {\edef\logfile{\filename@area\filename@base.log}}
  {\IfFileExists{\filename@area\filename@base.lis}
    {\edef\logfile{\filename@area\filename@base.lis}}
    {}}


%%
%% The example file goes here:
%%

\wmsg*{^^J%
Beispieldatei, die das Problem verdeutlicht:^^J%
============================================}

\ifx\samplefile\undefinedcommand
   \typeout{^^J%
      Beispieldatei \filename\space nicht gefunden.^^J%
      Bitte ergaenzen Sie das Beispiel in \jobname.msg.}
   \wmsg*{ < HIER TESTDATEI EINFUEGEN >}
   \pause
\else
   \copytomsg{\samplefile}
   \ifnum\linecount>60
    \typeout{%
^^J%
!!! Ihre Testdatei ist \the\linecount\space Zeilen lang.^^J%
!!! So grosse Testdateien erschweren die Ursachenfindun.:^^J%
!!! ^^J%
!!! Bitte verkleinern Sie Ihre Testdatei, soweit das ueberhaupt^^J%
!!! moeglich ist, so dass das Problem gerade noch auftritt.^^J}
   \pause
   \fi
\fi


%%
%% The log file goes here:
%%
\wmsg*{^^J%
LOG-Datei vom LaTeX-Lauf der Beispieldatei:^^J%
===========================================}

\ifx\logfile\undefinedcommand
   \typeout{^^J%
      LOG-Datei \filename@area\filename@base.log nicht gefunden.^^J%
      Bitte ergaenzen Sie die LOG-Datei in \jobname.msg.}
   \wmsg*{ < HIER LOG ZUR TESTDATEI EINFUEGEN >}
   \pause
\else
   \copytomsg{\logfile}
\fi

\fi


%%
%% Closing Banner.
%%
\typeout{^^J%
============================================================}

\ifinteractive
 \typeout{^^J%
   Weiteren Aenderungen koennen sie direkt in der Datei^^J%
   `\jobname.msg' mit Hilfe eines Editors vornehmen.}
\else
 \typeout{^^J%
   Das Formular fuer die Erstellung der Meldung wurde in die^^J%
   Datei `\jobname.msg' geschrieben, die Sie bitte mit Hilfe^^J%
   eines Editors ergaenzen, bevor Sie sie abschicken.}
\fi

\let\ifinteractivetrue\iftrue
\typeout{^^J%
  Wenn Sie ueber eMail verfuegen, so senden Sie `\jobname.msg'^^J%
  bitte an eine der folgenden Adressen:^^J%
  MausNet: \space Markus Kohm @ KA2 (maximal 48 KByte/Tag!)^^J%
  InterNet:\space unlf@rz.uni-karlsruhe.de^^J%
  Verwenden Sie dabei bitte als Betreff (Subject):^^J%
  \@spaces `KOMA-BUG:\space\synopsis'^^J%
^^J%
  Danke, dass Sie sich die Zeit genommen haben.}

\wmsg*{^^J%
============================================================^^J
   Ende der KOMA-Fehlermeldung.^^J%
============================================================}

%%
%% Close the .msg output stream.
%%
\immediate\closeout\msg

%%
%% This is the TeX primitive \end command.
%%
\@@end
